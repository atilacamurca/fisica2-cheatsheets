\section{Vetores}\label{vetores}

São representados por segmentos orientados e são caracterizados por

\begin{enumerate}
\def\labelenumi{\arabic{enumi}.}
\itemsep1pt\parskip0pt\parsep0pt
\item
  Módulo
\item
  Direção
\item
  Sentido
\end{enumerate}

\subsection{Soma de Vetores}\label{soma-de-vetores}

O resultado é um outro vetor com origem na origem do primeiro e
extremidade na extremidade do último.

\subsubsection{Propriedades da Soma}\label{propriedades-da-soma}

\paragraph{Comutatividade}\label{comutatividade}

\[
\vec{v} + \vec{w} = \vec{w} + \vec{v}
\]

\paragraph{Associatividade}\label{associatividade}

\[
\vec{u} + (\vec{v} + \vec{w}) = (\vec{u} + \vec{v}) + \vec{w}
\]

\paragraph{Existência do elemento
Neutro}\label{existuxeancia-do-elemento-neutro}

\[
\vec{v} + 0 = 0 + \vec{v} = \vec{v}
\]

\paragraph{Existência do elemento inverso da
soma}\label{existuxeancia-do-elemento-inverso-da-soma}

\[
\vec{v} + (-\vec{v}) = 0
\]

\subsection{Multiplicação de um vetor por um
escalar}\label{multiplicauxe7uxe3o-de-um-vetor-por-um-escalar}

Seja $\alpha$ um número real não-nulo e $\vec{v}$ um vetor. Dizemos que
$\alpha \vec{v}$:

\begin{enumerate}
\def\labelenumi{\arabic{enumi}.}
\itemsep1pt\parskip0pt\parsep0pt
\item
  Módulo: $|\alpha|$
\item
  Direção: mesma de $\vec{v}$
\item
  Sentido:

  \begin{itemize}
  \itemsep1pt\parskip0pt\parsep0pt
  \item
    mesmo sentido de $\vec{v}$ se $\alpha > 0$
  \item
    sentido oposto caso contrário
  \end{itemize}
\end{enumerate}

\subsubsection{Propriedades da multiplicação por
escalar}\label{propriedades-da-multiplicauxe7uxe3o-por-escalar}

\paragraph{Associatividade}\label{associatividade-1}

\[
\alpha (\beta \vec{v}) = (\alpha \beta) \vec{v}
\]

\paragraph{Distributividade}\label{distributividade}

\[
\alpha (\vec{v} + \vec{w}) = \alpha \vec{v} + \alpha \vec{w}
\]

\[
(\alpha + \beta) \vec{v} = \alpha \vec{v} + \beta \vec{v}
\]

\subsection{Produto Escalar}\label{produto-escalar}

Também conhecido como Produto Interno, é denotado por
$\vec{v} \cdot \vec{w}$, e definido por

\[
\vec{v} \cdot \vec{w} = v_1 w_1 + v_2 w_2
\]

Quando $\vec{v}$ e $\vec{w}$ são vetores no plano, e

\[
\vec{v} \cdot \vec{w} = v_1 w_1 + v_2 w_2 + v_3 w_3
\]

Quando $\vec{v}$ e $\vec{w}$ são vetores no espaço.

\subsubsection{Teorema}\label{teorema}

\[
\vec{v} \cdot \vec{w} = |\vec{v}| |\vec{w}| \cos \theta
\]

Onde $\theta$ é o ângulo entre estes vetores.

Daí temos também que

\[
\cos \theta = \frac{\vec{v} \cdot \vec{w}}{|\vec{v}| |\vec{w}|}
\]

\subsubsection{Propriedades do Produto
Escalar}\label{propriedades-do-produto-escalar}

\paragraph{Comutatividade}\label{comutatividade-1}

\[
\vec{v} \cdot \vec{w} = \vec{w} \cdot \vec{v}
\]

\paragraph{Distributividade}\label{distributividade-1}

\[
\vec{u} \cdot (\vec{v} + \vec{w}) = \vec{u} \cdot \vec{v} + \vec{u} \cdot \vec{w}
\]

\paragraph{Multiplicação por
escalar}\label{multiplicauxe7uxe3o-por-escalar}

\[
\alpha (\vec{v} \cdot \vec{w}) = (\alpha \vec{v}) \cdot \vec{w} = \vec{v} \cdot (\alpha \vec{w})
\]

\subsection{Produto Vetorial}\label{produto-vetorial}

É denotado por $\vec{v} \times \vec{w}$, e definido por

\[
|\vec{v} \times \vec{w}| = |\vec{v}| |\vec{w}| \sin \theta
\]

Direção: perpendicular ao plano determinado por $\vec{v}$ e $\vec{w}$

\subsubsection{Propriedades do Produto
Vetorial}\label{propriedades-do-produto-vetorial}

\paragraph{Anticomutatividade}\label{anticomutatividade}

\[
\vec{v} \times \vec{w} = - \vec{w} \times \vec{v}
\]

\paragraph{Distributividade}\label{distributividade-2}

\[
\vec{u} \times (\vec{v} + \vec{w}) = \vec{u} \times \vec{v} + \vec{u} \times \vec{w}
\]

\paragraph{Multiplicação por
escalar}\label{multiplicauxe7uxe3o-por-escalar-1}

\[
\alpha (\vec{v} \times \vec{w}) = (\alpha \vec{v}) \times \vec{w} = \vec{v} \times (\alpha \vec{w})
\]

\subsubsection{Teorema 1}\label{teorema-1}

Sejam $\vec{v} = (v_1, v_2, v_3)$ e $\vec{w} = (w_1, w_2, w_3) $. Então

\[
\vec{v} \times \vec{w}
= \det
\begin{vmatrix}
i & j & k \\ 
v_1 & v_2 & v_3 \\ 
w_1 & w_2 & w_3
\end{vmatrix}
\]
